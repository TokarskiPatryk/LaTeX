
\documentclass[a4paper]{article}

%% Created with wxMaxima 22.04.0

\setlength{\parskip}{\medskipamount}
\setlength{\parindent}{0pt}
\usepackage{iftex}
\ifPDFTeX
  % PDFLaTeX or LaTeX 
  \usepackage[utf8]{inputenc}
  \usepackage{polski}
  \DeclareUnicodeCharacter{00B5}{\ensuremath{\mu}}
\else
  %  XeLaTeX or LuaLaTeX
  \usepackage{fontspec}
\fi
\usepackage{graphicx}
\usepackage{color}
\usepackage{amsmath}
\usepackage{grffile}
\usepackage{dsfont}
\usepackage{ifthen}
\newsavebox{\picturebox}
\newlength{\pictureboxwidth}
\newlength{\pictureboxheight}
\newcommand{\includeimage}[1]{
    \savebox{\picturebox}{\includegraphics{#1}}
    \settoheight{\pictureboxheight}{\usebox{\picturebox}}
    \settowidth{\pictureboxwidth}{\usebox{\picturebox}}
    \ifthenelse{\lengthtest{\pictureboxwidth > .95\linewidth}}
    {
        \includegraphics[width=.95\linewidth,height=.80\textheight,keepaspectratio]{#1}
    }
    {
        \ifthenelse{\lengthtest{\pictureboxheight>.80\textheight}}
        {
            \includegraphics[width=.95\linewidth,height=.80\textheight,keepaspectratio]{#1}
            
        }
        {
            \includegraphics{#1}
        }
    }
}
\newlength{\thislabelwidth}
\DeclareMathOperator{\abs}{abs}

\definecolor{labelcolor}{RGB}{100,0,0}
%%%%%%%%%%%%%%%%%%%%
\author{Patryk Tokarski}
\title{Projekt na zaliczenie pakietu MAXIMA}


\begin{document}
\maketitle
\newpage


{\Large Nasza funkcja ma postać:}
\begin{equation*}
	f(x)=\sqrt[5]{x^5+3x^4-11x^3-27x^2+10x+26}
\end{equation*}

\noindent
%%%%%%%%
%% INPUT:
\begin{minipage}[t]{4.000000em}\color{red}\bfseries
(\% i1)	
\end{minipage}
\begin{minipage}[t]{\textwidth}\color{blue}
f(x):=(x\^\ 5+3*x\^\ 4-11*x\^\ 3-27*x\^\ 2+10*x+26)\^\ (1/5);
\end{minipage}

\section{Analiza funkcji} 
\subsection{Dziedzina funkcji}
Dziedziną funkcji jest cały zbiór $\mathds{R}$
%TODO
\subsection{Granice funkcji na krańcach dziedziny}

\[
	\lim_{x\to +\infty} f(x) = \lim_{x\to +\infty} \sqrt[5]{x^5+3x^4-11x^3-27x^2+10x+26} 	
\]
\[
	\lim_{x\to +\infty} f(x) = \sqrt[5]{\lim_{x\to +\infty} (x^5+3x^4-11x^3-27x^2+10x+26) } = +\infty
\]
Postępując analogicznie otrzymujemy $-\infty$ dla $x \to -\infty$.	
\subsection{Asymptoty wykresu funkcji}
\begin{itemize}
	\item Asymptota pionowa
		Z racji tego, że nasza funkcja jest ciągła to nie ma ona asymptot pionowych.
	\item Asymptoty ukośne 
		Przypomnijmy, że funkcja $f(x)$ ma asymptotę ukośną $y=Ax+B$ gdy \[\lim_{x \to +\infty} \frac{f(x)}{x} = A\]  \[\lim_{x\to +\infty}f(x)-Ax=B\]

\end{itemize}

\noindent
%%%%%%%%
%% INPUT:
\begin{minipage}[t]{4.000000em}\color{red}\bfseries
(\% i2)	
\end{minipage}
\begin{minipage}[t]{\textwidth}\color{blue}
limit(f(x)/x,x,inf);
\end{minipage}
%%%% OUTPUT:
\[\displaystyle \tag{\% o2} 
1\mbox{}
\]
%%%%%%%%%%%%%%%%


\noindent
%%%%%%%%
%% INPUT:
\begin{minipage}[t]{4.000000em}\color{red}\bfseries
(\% i3)	
\end{minipage}
\begin{minipage}[t]{\textwidth}\color{blue}
limit(f(x)-1*x,x,inf);
\end{minipage}
%%%% OUTPUT:
\[\displaystyle \tag{\% o3} 
\frac{3}{5}\mbox{}
\]
%%%%%%%%%%%%%%%%
Jak widzimy nasze $A=1$ i $B= \frac{3}{5}$\\
Czyli funkcja posiada asymptotę ukośną, która ma wzór $y = x+\frac{3}{5}$.
\section{Analiza pierwszej pochodnej}
Liczymy pochodną przy pomocy funkcji diff:\\
\noindent
%%%%%%%%
%% INPUT:
\begin{minipage}[t]{4.000000em}\color{red}\bfseries
(\% i4)	
\end{minipage}
\begin{minipage}[t]{\textwidth}\color{blue}
f\_1(x):=''(diff(f(x),x,1));
\end{minipage}
%%%% OUTPUT:
\[\displaystyle \tag{\% o4} 
\operatorname{f\_ 1}(x)\operatorname{:=}\frac{5 {{x}^{4}}+12 {{x}^{3}}-33 {{x}^{2}}-54 x+10}{5 {{\left( {{x}^{5}}+3 {{x}^{4}}-11 {{x}^{3}}-27 {{x}^{2}}+10 x+26\right) }^{\frac{4}{5}}}}\mbox{}
\]
%%%%%%%%%%%%%%%%

\subsection{Dziedzina pierwszej pochodnej}
Liczymy kiedy zeruje się mianownik pierwszej pochodnej:


\noindent
%%%%%%%%
%% INPUT:
\begin{minipage}[t]{4.000000em}\color{red}\bfseries
(\% i5)	
\end{minipage}
\begin{minipage}[t]{\textwidth}\color{blue}
realroots(x\^\ 5+3*x\^\ 4-11*x\^\ 3-27*x\^\ 2+10*x+26=0);
\end{minipage}
%%%% OUTPUT:
\[\displaystyle \tag{\% o5} 
\operatorname{[}x=-\frac{134533779}{33554432}\operatorname{,}x=-\frac{64689803}{33554432}\operatorname{,}x=-
\frac{36515359}{33554432}\operatorname{,}x=\frac{34653921}{33554432}\operatorname{,}x=\frac{100421725}{33554432}\operatorname{]}\mbox{}
\]
%%%%%%%%%%%%%%%%

Aby ułatwić dalsze operacje tworze zmienne t którymi oznaczę punkty wyłączone z dziedziny.\\ 
\noindent
%%%%%%%%
%% INPUT:
\begin{minipage}[t]{4.000000em}\color{red}\bfseries
(\% i7)	
\end{minipage}
\begin{minipage}[t]{\textwidth}\color{blue}
t1:-134533779/33554432\$\\
float(t1);
\end{minipage}
%%%% OUTPUT:
\[\displaystyle \tag{\% o7} 
-4.009419053792953\mbox{}
\]
%%%%%%%%%%%%%%%%


\noindent
%%%%%%%%
%% INPUT:
\begin{minipage}[t]{4.000000em}\color{red}\bfseries
(\% i9)	
\end{minipage}
\begin{minipage}[t]{\textwidth}\color{blue}
t2:-64689803/33554432\$\\
float(t2);
\end{minipage}
%%%% OUTPUT:
\[\displaystyle \tag{\% o9} 
-1.927906364202499\mbox{}
\]
%%%%%%%%%%%%%%%%


\noindent
%%%%%%%%
%% INPUT:
\begin{minipage}[t]{4.000000em}\color{red}\bfseries
(\% i11)	
\end{minipage}
\begin{minipage}[t]{\textwidth}\color{blue}
t3:-36515359/33554432\$\\
float(t3);
\end{minipage}
%%%% OUTPUT:
\[\displaystyle \tag{\% o11} 
-1.088242501020432\mbox{}
\]
%%%%%%%%%%%%%%%%


\noindent
%%%%%%%%
%% INPUT:
\begin{minipage}[t]{4.000000em}\color{red}\bfseries
(\% i13)	
\end{minipage}
\begin{minipage}[t]{\textwidth}\color{blue}
t4:34653921/33554432\$\\
float(t4);
\end{minipage}
%%%% OUTPUT:
\[\displaystyle \tag{\% o13} 
1.032767325639725\mbox{}
\]
%%%%%%%%%%%%%%%%


\noindent
%%%%%%%%
%% INPUT:
\begin{minipage}[t]{4.000000em}\color{red}\bfseries
(\% i15)	
\end{minipage}
\begin{minipage}[t]{\textwidth}\color{blue}
t5:100421725/33554432\$\\
float(t5);
\end{minipage}
%%%% OUTPUT:
\[\displaystyle \tag{\% o15} 
2.992800623178482\mbox{}
\]
%%%%%%%%%%%%%%%%
Zatem dziedzina pochodnej to $\mathds{R}/\{t1,t2,t3,t4,t5\}$

\subsection{Miejsca zerowe pierwszej pochodnej}

Obliczamy je przy pomocy odpowiedniej funkcji i zapisujemy jako kolejne zmienne x:

\noindent
%%%%%%%%
%% INPUT:
\begin{minipage}[t]{4.000000em}\color{red}\bfseries
(\% i16)	
\end{minipage}
\begin{minipage}[t]{\textwidth}\color{blue}
realroots(f\_1(x));
\end{minipage}
%%%% OUTPUT:
\[\displaystyle \tag{\% o16} 
\operatorname{[}x=-\frac{112567015}{33554432}\operatorname{,}x=-\frac{51220217}{33554432}\operatorname{,}x=\frac{5667285}{33554432}\operatorname{,
}x=\frac{77589309}{33554432}\operatorname{]}\mbox{}
\]
%%%%%%%%%%%%%%%%


\noindent
%%%%%%%%
%% INPUT:
\begin{minipage}[t]{4.000000em}\color{red}\bfseries
(\% i20)	
\end{minipage}
\begin{minipage}[t]{\textwidth}\color{blue}
x1:-112567015/33554432\$\\
x2:-51220217/33554432\$\\
x3:5667285/33554432\$\\
x4:77589309/33554432\$
\end{minipage}

\noindent%

\subsection{Przedziały monotoniczności funkcji}


\noindent
%%%%%%%%
%% INPUT:
\begin{minipage}[t]{4.000000em}\color{red}\bfseries
(\% i21)	
\end{minipage}
\begin{minipage}[t]{\textwidth}\color{blue}
limit(f\_1(x),\ x,\ minf);
\end{minipage}
%%%% OUTPUT:
\[\displaystyle \tag{\% o21} 
1\mbox{}
\]
%%%%%%%%%%%%%%%%
Wartość pochodnej w minus nieskończoności to 1 więc:
$$f'(x)>0 \mbox{ wtt gdy } x \in (-\infty,x_1)\cup(x_2,x_3)\cup(x_4,+\infty) \\$$
$$f'(x)<0 \mbox{ wtt gdy } x \in (x_1,x_2) \cup (x_3,x_4)$$
\subsection{Ekstrema funkcji}
Ekstrema minima lokalne są w punktach: $x_2, x_4$.\\
Ekstrema maksima lokalne są w punktach: $x_1, x_3$.\\
By określić ekstrema globalne liczymy wartosci funkcji w wyżej wymienionych punktach:

\noindent
%%%%%%%%
%% INPUT:
\begin{minipage}[t]{4.000000em}\color{red}\bfseries
(\% i22)	
\end{minipage}
\begin{minipage}[t]{\textwidth}\color{blue}
float(f(x1));
\end{minipage}
%%%% OUTPUT:
\[\displaystyle \tag{\% o22} 
2.260039727389741\mbox{}
\]
%%%%%%%%%%%%%%%%


\noindent
%%%%%%%%
%% INPUT:
\begin{minipage}[t]{4.000000em}\color{red}\bfseries
(\% i23)	
\end{minipage}
\begin{minipage}[t]{\textwidth}\color{blue}
float(f(x2));
\end{minipage}
%%%% OUTPUT:
\[\displaystyle \tag{\% o23} 
-1.382587454924842\mbox{}
\]
%%%%%%%%%%%%%%%%


\noindent
%%%%%%%%
%% INPUT:
\begin{minipage}[t]{4.000000em}\color{red}\bfseries
(\% i24)	
\end{minipage}
\begin{minipage}[t]{\textwidth}\color{blue}
float(f(x3));
\end{minipage}
%%%% OUTPUT:
\[\displaystyle \tag{\% o24} 
1.931293053098893\mbox{}
\]
%%%%%%%%%%%%%%%%


\noindent
%%%%%%%%
%% INPUT:
\begin{minipage}[t]{4.000000em}\color{red}\bfseries
(\% i25)	
\end{minipage}
\begin{minipage}[t]{\textwidth}\color{blue}
float(f(x4));
\end{minipage}
%%%% OUTPUT:
\[\displaystyle \tag{\% o25} 
-2.398446670693105\mbox{}
\]
%%%%%%%%%%%%%%%%

Jak widać ekstremum maksimum globalne jest w punkcie $x_1$, a ekstremum minimum globalne w punkcie $x_4$.
\section{Analiza drugiej pochodnej}
Najpierw obliczymy z pomocą Maximy drugą pochodną 


\noindent
%%%%%%%%
%% INPUT:
\begin{minipage}[t]{4.000000em}\color{red}\bfseries
(\% i26)	
\end{minipage}
\begin{minipage}[t]{\textwidth}\color{blue}
f\_2(x):=''(diff(f(x),x,2));
\end{minipage}
%%%% OUTPUT:
\[\displaystyle \tag{\% o26} 
\operatorname{f\_ 2}(x)\operatorname{:=}\frac{20 {{x}^{3}}+36 {{x}^{2}}-66 x-54}{5 {{\left( {{x}^{5}}+3 {{x}^{4}}-11 {{x}^{3}}-27 {{x}^{2}}+10 x+26\right) }^{\frac{4}{5}}}}-
\frac{4 {{\left( 5 {{x}^{4}}+12 {{x}^{3}}-33 {{x}^{2}}-54 x+10\right) }^{2}}}{25 {{\left( {{x}^{5}}+3 {{x}^{4}}-11 {{x}^{3}}-27 {{x}^{2}}+10 x+26\right) }^{\frac{9}{5}}}}\mbox{}
\]
%%%%%%%%%%%%%%%%
Po uproszczeniu dostajemy, że nasza druga pochodna jest równa:


\noindent
%%%%%%%%
%% INPUT:
\begin{minipage}[t]{4.000000em}\color{red}\bfseries
(\% i27)	
\end{minipage}
\begin{minipage}[t]{\textwidth}\color{blue}
f\_2\_simple(x):=''(ratsimp(f\_2(x)));
\end{minipage}
%%%% OUTPUT:
\[
\operatorname{f\_ 2\_ simple}(x)\operatorname{:=}-\operatorname{(}{{\left( {{x}^{5}}+3 {{x}^{4}}-11 {{x}^{3}}-27 {{x}^{2}}+10 x+26\right) }^{\frac{1}{5}}}\]
\[\operatorname{(}146 {{x}^{6}}+612 {{x}^{5}}+
612 {{x}^{4}}-1064 {{x}^{3}}+354 {{x}^{2}}+6960 x+7420\operatorname{)}\operatorname{)}\]
\[/\operatorname{(}25 {{x}^{10}}+150 {{x}^{9}}-325 {{x}^{8}}-3000 {{x}^{7}}-525 {{x}^{6}}+17650 {{x}^{5}}+16625 {{x}^{4}}-27800 {{x}^{3}}-32600 {{x}^{2}}+13000 x+16900\operatorname{)}\mbox{}
\]
%%%%%%%%%%%%%%%%
\subsection{Dziedzina drugiej pochodnej}
Liczymy kiedy mianownik się zeruje:


\noindent
%%%%%%%%
%% INPUT:
\begin{minipage}[t]{4.000000em}\color{red}\bfseries
(\% i28)	
\end{minipage}
\begin{minipage}[t]{\textwidth}\color{blue}
realroots(25*x\^\ 10+150*x\^\ 9-325*x\^\ 8-3000*x\^\ 7-525*x\^\ 6+17650*x\^\ 5+16625*x\^\ 4-27800*x\^\ 3-32600*x\^\ 2+13000*x+16900);
\end{minipage}
%%%% OUTPUT:
\[\displaystyle \tag{\% o28} 
\operatorname{[}x=-\frac{134533779}{33554432}\operatorname{,}x=-\frac{64689803}{33554432}\operatorname{,}x=-
\frac{36515359}{33554432}\operatorname{,}x=\frac{34653921}{33554432}\operatorname{,}x=\frac{100421725}{33554432}\operatorname{]}\mbox{}
\]
%%%%%%%%%%%%%%%%
Do dziedziny nie należą właśnie te punkty. Są to punkty wcześniej oznaczone przez t1,t2,... . Oznacza to, że dziedzina drugiej pochodnej jest taka sama jak pierwszej.
\subsection{Miejsca zerowe drugiej pochodnej}

\noindent
%%%%%%%%
%% INPUT:
\begin{minipage}[t]{4.000000em}\color{red}\bfseries
(\% i29)	
\end{minipage}
\begin{minipage}[t]{\textwidth}\color{blue}
allroots(f\_2(x));
\end{minipage}
%%%% OUTPUT:
\[\displaystyle \mbox{}\\\mbox{%default
allroots: expected a polynomial; found }\ensuremath{\mathrm{errexp1
}}\mbox{%error
 -- an error. To debug this try: debugmode(true);}\mbox{}
\]
%%%%%%%%%%%%%%%%
Otrzymaliśmy błąd, więc druga pochodna nie ma miejsc zerowych. 
\subsection{Przedziały stałego znaku drugiej pochodnej}

Skoro druga pochodna nie ma miejsc zerowych, to w każdym przedziale dziedziny sprawdzimy wartość funkcji w dowolnym punkcie, dzięki czemu będziemy wiedzieć jaki jest znak wartości drugiej pochodnej w tym przedziale.

\begin{itemize}
	\item 1 przedział (-oo,t1):

\noindent
%%%%%%%%
%% INPUT:
\begin{minipage}[t]{4.000000em}\color{red}\bfseries
(\% i30)	
\end{minipage}
\begin{minipage}[t]{\textwidth}\color{blue}
float(f\_2(-5));
\end{minipage}
%%%% OUTPUT:
\[\displaystyle \tag{\% o30} 
0.374451897594516\mbox{}
\]
%%%%%%%%%%%%%%%%

\item 2 przedział (t1,t2):


\noindent
%%%%%%%%
%% INPUT:
\begin{minipage}[t]{4.000000em}\color{red}\bfseries
(\% i31)	
\end{minipage}
\begin{minipage}[t]{\textwidth}\color{blue}
float(f\_2(-2.5));
\end{minipage}
%%%% OUTPUT:
\[\displaystyle \tag{\% o31} 
-1.163066666628178\mbox{}
\]
%%%%%%%%%%%%%%%%
\item 3 przedział (t2,t3):


\noindent
%%%%%%%%
%% INPUT:
\begin{minipage}[t]{4.000000em}\color{red}\bfseries
(\% i32)	
\end{minipage}
\begin{minipage}[t]{\textwidth}\color{blue}
float(f\_2(-1.5));
\end{minipage}
%%%% OUTPUT:
\[\displaystyle \tag{\% o32} 
3.233832698501276\mbox{}
\]
%%%%%%%%%%%%%%%%
\item 4 przedział (t3,t4):


\noindent
%%%%%%%%
%% INPUT:
\begin{minipage}[t]{4.000000em}\color{red}\bfseries
(\% i33)	
\end{minipage}
\begin{minipage}[t]{\textwidth}\color{blue}
float(f\_2(0));
\end{minipage}
%%%% OUTPUT:
\[\displaystyle \tag{\% o33} 
-0.842387415494661\mbox{}
\]
%%%%%%%%%%%%%%%%
\item 5 przedział (t4,t5):


\noindent
%%%%%%%%
%% INPUT:
\begin{minipage}[t]{4.000000em}\color{red}\bfseries
(\% i34)	
\end{minipage}
\begin{minipage}[t]{\textwidth}\color{blue}
float(f\_2(2));
\end{minipage}
%%%% OUTPUT:
\[\displaystyle \tag{\% o34} 
1.011263415849287\mbox{}
\]
%%%%%%%%%%%%%%%%
\item 6 przedział (t5,+oo):


\noindent
%%%%%%%%
%% INPUT:
\begin{minipage}[t]{4.000000em}\color{red}\bfseries
(\% i35)	
\end{minipage}
\begin{minipage}[t]{\textwidth}\color{blue}
float(f\_2(4));
\end{minipage}
%%%% OUTPUT:
\[\displaystyle \tag{\% o35} 
-0.3875940735998362\mbox{}
\]
%%%%%%%%%%%%%%%%

\end{itemize}



\begin{equation*}
	f''(x)>0 \mbox{ gdy } x \in (-\infty,t_1) \cup (t_2,t_3) \cup (t_4,t_5)
\end{equation*}
\begin{equation*}
	f''(x)<0 \mbox{ gdy } x \in (t_1,t_2) \cup (t_3,t_4) \cup (t_5,+\infty)
\end{equation*}

Uzyskaliśmy dzięki temu informację, że nasza krzywa jest wypukła, gdy x nalezy do przedziałów, które dla drugiej pochodnej są większe od zera, a wklęsła gdy x należy do tych przedziałów, które dla drugiej pochodnej są mniejsze od zera.


\pagebreak
\subsection{Punkty przegięcia wykresu}
Punktami przegięcia w naszym przypadku będą punkty wyłączone z dziedziny pochodnych. Przypomnijmy, że są to punkty t1, t2, t3, t4, t5.

\section{Tabela zmienności funkcji}
Tabelę podzieliłem na dwie części, ponieważ w przeciwnym przypadku nie zmieściłaby się.

%%%%%%%%%%%%%%%% TABELA
\begin{table}[h]
	\label{tab:my-table1}
	\begin{tabular}{l|llclclclclc}
		       & \multicolumn{1}{c}{$+\infty$} &   & t1                                                  &   & \multicolumn{1}{l}{x1}                                  &   & \multicolumn{1}{l}{t2}                              &   & \multicolumn{1}{l}{x2}                                   &   & \multicolumn{1}{l}{t3}                              \\ \hline
		f'(x)  & \multicolumn{1}{c}{}    & + & *                                                   & + & 0                                                       & - & *                                                   & - & 0                                                        & + & *                                                   \\
		f''(x) &                         & + & *                                                   & - & -                                                       & - & *                                                   & + & +                                                        & + & *                                                   \\
		f(x)   & $+\infty$                     &   & \begin{tabular}[c]{@{}c@{}}pp\\ (t1,0)\end{tabular} &   & \begin{tabular}[c]{@{}c@{}}MAX\\ (x1,2.26)\end{tabular} &   & \begin{tabular}[c]{@{}c@{}}pp\\ (t2,0)\end{tabular} &   & \begin{tabular}[c]{@{}c@{}}min\\ (x2,-1.38)\end{tabular} &   & \begin{tabular}[c]{@{}c@{}}pp\\ (t3,0)\end{tabular}
	\end{tabular}
\end{table}
\begin{table}[h]

	\label{tab:my-table}
	\begin{tabular}{l|lclclclclc}
		       &   & \multicolumn{1}{l}{x3}                                  &   & \multicolumn{1}{l}{t4}                              &   & \multicolumn{1}{l}{x4}                                  &   & \multicolumn{1}{l}{t5}                              &   & \multicolumn{1}{l}{$+\infty$} \\ \hline
		f'(x)  & + & 0                                                       & - & *                                                   & - & 0                                                       & + & *                                                   & + &                         \\
		f''(x) & - & -                                                       & - & *                                                   & + & +                                                       & + & *                                                   & - &                         \\
		f(x)   &   & \begin{tabular}[c]{@{}c@{}}max\\ (x3,1.93)\end{tabular} &   & \begin{tabular}[c]{@{}c@{}}pp\\ (t4,0)\end{tabular} &   & \begin{tabular}[c]{@{}c@{}}MIN\\ (x4,-2.4)\end{tabular} &   & \begin{tabular}[c]{@{}c@{}}pp\\ (t5,0)\end{tabular} &   & $+\infty$                    

	\end{tabular}
	\caption{Tabela zmienności funkcji}

\end{table}
	* - punkt wyłączony z dziedziny;\\
	+/- - znak wartości funkcji w przedziale
%%%%%%%%%%%%%%%%%%% TABELA
\pagebreak
\section{Wykres funkcji}
Wykres stworzono przy pomocy programu Maxima.

\noindent
%%%%%%%%
%% INPUT:
\begin{minipage}[t]{4.000000em}\color{red}\bfseries
(\% i36)	
\end{minipage}
\begin{minipage}[t]{\textwidth}\color{blue}
wxplot2d(f(x),\ [x,-10,10],[y,-10,10],[axes,\ solid],\\
\ \ \ \ [grid2d,\ true],[xtics,\ 1],[ytics,\ 1],\\
\ \ \ \ [title,\ "Wykres\ funkcji\ f(x)"],\ [xlabel,\ "x"],\\
\ \ \ \ [ylabel,\ "f(x)"])\$\\

\end{minipage}
%%%% OUTPUT:

\[ 
\includegraphics[width=.95\linewidth,height=.80\textheight,keepaspectratio]{projekt_img/projekt_1}\mbox{}
\]
%%%%%%%%%%%%%%%%

\end{document}
